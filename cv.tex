\documentclass[11pt,a4paper,sans]{moderncv}        % possible options include font size ('10pt', '11pt' and '12pt'), paper size ('a4paper', 'letterpaper', 'a5paper', 'legalpaper', 'executivepaper' and 'landscape') and font family ('sans' and 'roman')

\moderncvstyle{classic}                             % style options are 'casual' (default), 'classic', 'oldstyle' and 'banking'
\moderncvcolor{green}                               % color options 'blue' (default), 'orange', 'green', 'red', 'purple', 'grey' and 'black'
%\renewcommand{\familydefault}{\sfdefault}         % to set the default font; use '\sfdefault' for the default sans serif font, '\rmdefault' for the default roman one, or any tex font name
%\nopagenumbers{}                                  % uncomment to suppress automatic page numbering for CVs longer than one page

% character encoding
\usepackage[utf8]{inputenc}                       % if you are not using xelatex ou lualatex, replace by the encoding you are using

% adjust the page margins
\usepackage[scale=0.75]{geometry}
\AtBeginDocument{\hypersetup{colorlinks,urlcolor=color1}}
%\setlength{\hintscolumnwidth}{3cm}                % if you want to change the width of the column with the dates
%\setlength{\makecvtitlenamewidth}{10cm}           % for the 'classic' style, if you want to force the width allocated to your name and avoid line breaks. be careful though, the length is normally calculated to avoid any overlap with your personal info; use this at your own typographical risks...

% personal data
\name{Pavel}{Shevchuk} 
\phone[mobile]{+7~(903)~217~56~96}
\email{pashevchuk@edu.hse.ru}
\social[github]{VisualPaul}
\social[linkedin]{pshevchuk}
\extrainfo{codeforces profile: \href{http://codeforces.com/profile/pshevchuk}{pshevchuk} \\
  Moscow, Russia}
\photo[64pt][0.4pt]{picture}                       % optional, remove / comment the line if not wanted; '64pt' is the height the picture must be resized to, 0.4pt is the thickness of the frame around it (put it to 0pt for no frame) and 'picture' is the name of the picture file

% to show numerical labels in the bibliography (default is to show no labels); only useful if you make citations in your resume
%\makeatletter
%\renewcommand*{\bibliographyitemlabel}{\@biblabel{\arabic{enumiv}}}
%\makeatother
%\renewcommand*{\bibliographyitemlabel}{[\arabic{enumiv}]}% CONSIDER REPLACING THE ABOVE BY THIS

% bibliography with mutiple entries
%\usepackage{multibib}
%\newcites{book,misc}{{Books},{Others}}
%----------------------------------------------------------------------------------
%            content
%----------------------------------------------------------------------------------
\begin{document}
\makecvtitle
\emph{Winner of competitive programming contests with mathematical skills looking for CS internship}
\section{Interests}
\cvitem{}{computer science(machine learning, algorithms), \newline{}competitive porgramming, mathematics}

\section{Education}
\cventry{2014--2018}{BS CS}{NRU Higher School of Economics}{Moscow}{}{}
\cventry{2016}{ADFOCS}{Summer school on distributed computing}{Saarbrücken}{}{}

\section{Technical experience}
\subsection{Programming languages}
\cvitem{proficiency}{C++, Python}
\cvitem{some expirience}{Java, C\#, Haskell, Assembly, Bash}
\subsection{Other technologies}
\cvitem{using every day}{git, Linux (grep/sed/perl/bash, emacs), \LaTeX , CMake, scikit, vowpal wabbit}
\cvitem{use from time to time}{svn, vim, HTML. Qt, Gtk, make}

\section{Achievements}
\cvitemwithcomment{2015}{Northeastern Europe Regional Contest participant}
{\href{https://icpc.baylor.edu/regionals/finder/moscow-subregional-2015/standings}{link}}
\cvitemwithcomment{2015}{Vekua cup award}
{\href{http://vekua.snarknews.info/index.cgi?data=macros/ind_ons&class=vekua2015&year=2015}{link}}
\cvitemwithcomment{2015}{VK wild-card prize - advanced to Round 3}
{\href{http://codeforces.com/vkcup2015}{link}}
\cvitemwithcomment{2014}{Open Olympiads in Programming 2014 - first award}{\href{https://olympiads.ru/zaoch/2013-14/final_results.shtml}{link}}                  
\cvitemwithcomment{2014,2013}{Finalist of Russian Olympiads in Informatics}
{\href{http://neerc.ifmo.ru/school/archive/2013-2014/ru-olymp-roi-2014-standings.html}{link 2014} \href{http://neerc.ifmo.ru/school/archive/2012-2013/ru-olymp-roi-2013-standings.html}{link 2015}}

\section{Work expierence}
\cventry{2016--present}{Computer Vision junior}{}{}{}{}
\cventry{2016}{Teaching school students programming in a summer school}{}{}{}{}
\cventry{2015--2016}{Undergraduate teaching assistant for  Computer Science 1st grade course}{}{}{}{}
\cventry{2014}{Construction engineer assistant}{}{}{}{}

\section{Projects}
\cventry{2014--2015}{Lisp interpreter}{}{}{}{My main achievement here is garbage collection, %
which really works.\newline{}%
Skills \& technologies used and improved:
\begin{itemize}
\item C++. Much of C++
\item valgrind. Without it debugging of this project would be a hell.
\item GNU MP. Very useful bignum library
\item CMake
\end{itemize}
\href{https://github.com/VisualPaul/lisp-interp}{this project on github}}

\cventry{2015}{last.fm classification}{}{}{}{Web project that finds out preferences of last.fm user %
and understand whether he or she will like given artist\newline{}%
Skills \& technologies used and improved:
\begin{itemize}
\item Python
\item Jinja2 + HTML, css, 
\item Linear classification
\item Google App Engine
\item Beaker cache
\item REST API
\item Asynchronious code
\end{itemize}
\href{https://github.com/cs-hse-projects/Lfm_class_shevchuk}{this project on github}\newline{}
\href{https://lastfm-classificator.appspot.com/}{this project in the real life}}
\cventry{2015}{Source code plagiarism detector}{}{}{}{Source code plagiarism detector for VK cup 2015 wild-card round 2 \newline{}
Skills \& technologies used:
\begin{itemize}
\item C++
\item Python
\item Levenstein distance
\item Polynomial hashing
\item Research skills
\end{itemize}
\href{https://github.com/VisualPaul/vk-wildcard}{this project on github}\newline{}}
\cventry{2016}{Post stamps search}{}{}{}{A problem was to find the same stamps from the database by picture. \newline{}
The project as whole is a Docker, which can be called via RESTful API. \newline{}
Skills \& technologies used:
\begin{itemize}
\item OpenCV
\item scikit-image
\item Django
\item RESTful API
\item Docker
\item Research skills
\end{itemize}}

\section{Languages}
\cvitem{Russian}{native speaker}
\cvitem{English}{professional working proficiency}
\cvitem{German}{basic knowledge}

\end{document}


